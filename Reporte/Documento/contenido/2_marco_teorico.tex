% ----------------------
  \chapter{Marco Teórico} 
% ----------------------
\label{C:antecedentes}

\section{Sistema de alarmas contra incendio}


\subsection{Tipos de fuego}

Cuando ocurre un incendio este se debe a la combustión de ciertos materiales, razón por la cual según la NFPA 72 y el cuerpo de bomberos de Costa Rica clasifica los tipos de fuego en: \cite{Bomberos}\\

\begin{itemize}
	
	\item \textbf{Clase A:} 
	
\begin{itemize}
	\item \textbf{Material combustible:} combustibles comunes como madera, tela, papel, caucho y plásticos.
\end{itemize}
	
	
	\item \textbf{Clase B:} 

\begin{itemize}
	\item \textbf{Material combustible:} líquidos y gases inflamables como aceites, grasas,
	alquitranes, base de pinturas y lacas.
\end{itemize}	


	\item \textbf{Clase C:} 

\begin{itemize}
	\item \textbf{Material combustible:} equipos eléctricos energizados.
\end{itemize}



	\item \textbf{Clase D:} 

\begin{itemize}
	\item \textbf{Material combustible:} metales como magnesio, titanio, zirconio, sodio, litio, potasio entre otros.
\end{itemize}

	\item \textbf{Clase K:} 

\begin{itemize}
	\item \textbf{Material combustible:} utensilios o materiales de cocina como aceites minerales, animales y grasas.
\end{itemize}

	
	
\end{itemize}


\subsection{Dispositivos de iniciación ó detección de incendio}


Son los dispositivos del sistema que se encargan de censar constantemente las condiciones del ambiente en busca de indicios de incendio. Estos dispositivos o sensores pueden ser direccionales o no, es decir que pueden indicar la locación o zona del incendio o bien solo activar el sistema. El fuego tiene ciertas características que pueden ser censadas y dar indicios de incendio, tales como humo, llama, calor, entre otros. 


\subsubsection{Detectores de humo}

Son aquellos dispositivos que detectan como indicio de incendio partículas de humo visible o invisible. Estos varían según su aplicación y tiempo de respuesta. (3.3.59.19)

\begin{itemize}
	
	\item \textbf{Detector por cámara de niebla:} utiliza un dispositivo fotoeléctrico para medir la densidad de una muestra de aire dentro de la recamara del sensor, si la densidad de la muestro original es variada debido a la presencia de partículas de humo el sensor entra en condición de alarma. (3.3.252.1)
	
	
	\item \textbf{Detector por ionización:} utiliza un material radiactivo para ionizar el aire entre dos electrodos, cuando existe presencia de partículas de humo estas causan que el flujo de iones decrezca y el sensor entre en condición de alarma si cumple con los parámetros establecidos. (3.3.252.2)
	
	
	\item \textbf{Detector por efecto fotoeléctrico de obstrucción:} utilizan una fuente de luz y un foto-receptor no enfocado, al entrar humo en el sensor este dispersa la luz produciendo que parte de esta llegue al foto-receptor quien evaluara dicha dispersión y dará condición de alarma si cumple con los parámetros establecidos. (3.3.252.3)
	
	\item \textbf{Detector por efecto fotoeléctrico de dispersión:} utilizan una fuente de luz y un foto-receptor enfocado, al entrar humo en el sensor este dispersa la luz produciendo que disminuya la cantidad de luz que llega al foto-receptor quien evaluara dicha dispersión y dará condición de alarma si cumple con los parámetros establecidos. (3.3.252.4)

	\item \textbf{Detector por imagen de video:} utiliza técnicas de análisis de imagen en tiempo real para detectar la presencia de humo. (3.3.252.5)
	
	\item \textbf{Detector por haz proyectado:} utiliza una fuente de luz, un foto-receptor y un espejo, cuando hay presencia de partículas de humo entre el haz de luz y el espejo, este dispersa la luz que llega al foto-receptor produciendo la señal de alarma. (3.3.59.15)

	\item \textbf{Detector en ducto de aire acondicionado:} responde ante al censado de partículas de humo en el sistema de aire acondicionado. (17.7.4)	
	
	\newpage
	
	\item \textbf{Detector de muestreo de aire:} consiste en una red de tuberías que van desde el detector hasta las áreas a proteger. El detector aspira aire de la zona a proteger y lo hace correr a través de la red de tuberías pasando por varios puestos de muestreo para detección de humo. (3.3.59.1)
		

\end{itemize}


\subsubsection{Detectores de gas por fuego}

Consiste en dispositivo que detecta gases producidos por fuego como $CO_{2}$, $CO$, $N_{2}$, $H_{2}$, entre otros. (3.3.59.6)


\subsubsection{Detectores de energía radiante}

Consiste en un dispositivo capaz de censar la energía radiante como ultravioleta, visible o infrarrojo. (3.3.59.16)


\begin{itemize}
	
	\item \textbf{Detector de llama:} dispositivo capaz de censar la energía radiante producida por grandes llamas. (3.3.59.8)
	
	\item \textbf{Detector de chispas y brasas:} dispositivo capaz de censar la energía radiante producida por chispas y brasas. Generalmente utilizado en lugares oscuros y en el rango de infrarrojo. (3.3.59.8)
	
\end{itemize}



\subsubsection{Detectores de calor}

Consiste en un dispositivo capaz de censar la temperatura, la taza de cambio de la temperatura o ambos de un lugar determinado. (3.3.59.9)


\begin{itemize}
	
	\item \textbf{Detector por conductividad eléctrica:} utiliza una resistencia que varía en función de la temperatura. (3.3.59.5)
	
	\item \textbf{Detector de temperatura fija:} responde cuando su elemento operativo se calienta a una temperatura determinada. (3.3.59.7)
	
	\item \textbf{Detector con tubería de tasa de incremento neumático:} consiste en una serie de tuberías de pequeñas de cobre que se instalan en el techo y los muros. El tubo termina en un detector que contiene diafragmas y contactos configurados para actuar a una presión predeterminada. El sistema es lo suficientemente robusto para aceptar los pequeños cambios de temperaturas normales. (3.3.59.14)
	
	\item \textbf{Detector de tasa de compensación:} responde cuando la temperatura del aire que rodea el dispositivo alcanza un nivel determinado. (3.3.59.17)
	
	\item \textbf{Detector de incremento:} responde cuando la tasa de incremento en la temperatura supera un valor determinado. (3.3.59.18)
	
\end{itemize}
 
 
\subsubsection{Detectores de flujo} 
 
Consiste en un sensor que monitorea el flujo en la tubería de supresión de incendio, si hay un cambio en el flujo de la tubería por más de 90 segundos igual o superior al rociador más pequeño instalado, se envía señal de alarma. (17.12.2)
 

\subsubsection{Detectores multi-criterio}

Consiste en un dispositivo con múltiples sensores que responden por separado ante un estímulo físico como calor, humo, gases de combustión, entre otros. El detector envía una única señal de alarma, ya sea por la activación de uno de los sensores o varios de ellos. Este dispositivo tiene la capacidad de priorizar su censado según su aplicación. (3.3.59.11)   


\subsubsection{Estaciones manuales}

Es un dispositivo utilizado manualmente para iniciar la señal de alarma de incendio. (3.3.8.3)



\subsection{Dispositivos de notificación de incendio}

Consisten en dispositivos que por medio de luces, bocinas, táctil o mensajes escritos informan el estado de alarma y la ruta de evacuación a las personas dentro de la zona de riesgo. (3.3.160)

\subsubsection{Notificación audible}

Consiste en la notificación utilizando el sentido de la audición, generalmente mediante el uso de sirenas y  bocinas. (3.3.160.1)


\begin{itemize}
	
	\item \textbf{Notificación audible de salida:} utiliza el sentido de la audición con el fin de guiar a las personas en riego a la salida más cercana, se utiliza para rutas de evacuación o reubicación. (3.3.160.1.1)
	
	\item \textbf{Notificación audible de texto:} utiliza un mensaje pre-gravado para informar a las personas en riego el estado de alarma, además brinda indicaciones y medidas de seguridad. (3.3.160.1.2)
		
\end{itemize}


\subsubsection{Notificación táctil}

Consiste en un dispositivo que alerta por medio del sentido del tacto o la vibración. (3.3.160.2)

\subsubsection{Notificación visual}

Consiste en un dispositivo que alerta por medio del sentido de la vista. (3.3.160.3)


\begin{itemize}
	
	\item \textbf{Notificación visual de salida:} utiliza el sentido de la vista con el fin de guiar a las personas en riego a la salida más cercana, se utiliza para rutas de evacuación o reubicación, generalmente luces estroboscópicas. (3.3.160.3)
	
	\item \textbf{Notificación visual de texto:} utiliza un mensaje visual para informar a las personas en riego el estado de alarma, además brinda indicaciones y medidas de seguridad, típicamente monitores y pantallas. (3.3.160.3.1)
	
\end{itemize}



\subsection{Panel de control y fuente de poder}



%
%
%10.5.6.3* Capacity.
%
%10.5.6.3.1 The secondary power supply shall have sufficient
%capacity to operate the system under quiescent load (system
%operating in a nonalarm condition) for a minimum of
%24 hours and, at the end of that period, shall be capable of
%operating all alarm notification appliances used for evacuation
%or to direct aid to the location of an emergency for 5 minutes,
%unless otherwise permitted or required by the following:
%
%(1) Battery calculations shall include a 20 percent safety mar-
%gin to the calculated amp-hour rating.
%(2) The secondary power supply for in-building fire emer-
%gency voice/alarm communications service shall be ca-
%pable of operating the system under quiescent load for a
%minimum of 24 hours and then shall be capable of oper-
%ating the system during a fire or other emergency condi-
%tion for a period of 15 minutes at maximum connected
%load.
%(3) The secondary power supply capacity for supervising sta-
%tion facilities and equipment shall be capable of support-
%ing operations for a minimum of 24 hours.
%(4) The secondary power supply for high-power speaker ar-
%rays used for wide-area mass notification systems shall be
%in accordance with 24.4.3.4.2.2.
%(5) The secondary power supply for textual visible appliances
%shall be in accordance with 24.4.3.4.7.1.
%(6) The secondary power supply capacity for central control sta-
%tions of a wide-area mass notification systems shall be ca-
%pable of supporting operations for a minimum of 24 hours.
%(7) The secondary power supply for in-building mass notifica-
%tion systems shall be capable of operating the system un-
%der quiescent load for a minimum of 24 hours and then
%shall be capable of operating the system during emer-
%gency condition for a period of 15 minutes at maximum
%connected load.
%10.5.6.3.2 The secondary power supply capacity required
%shall include all power supply loads that are not automatically
%disconnected upon the transfer to secondary power supply.
%10.5.6.4 Secondary Power Operation.
%10.5.6.4.1 Operation on secondary power shall not affect the
%required performance of a system or supervising station facility,
%including alarm, supervisory, and trouble signals and indications.
%10.5.6.4.2 Systems operating on secondary power shall com-
%ply with Section 10.17.
%10.5.6.4.3 While operating on secondary power, audio ampli-
%fier monitoring shall comply with 10.17.2.1.2.
%10.5.7* Continuity of Power Supplies.
%10.5.7.1 The secondary power supply shall automatically pro-
%vide power to the protected premises system within 10 seconds
%whenever the primary power supply fails to provide the mini-
%mum voltage required for proper operation.
%10.5.7.2 The secondary power supply shall automatically pro-
%vide power to the supervising station facility and equipment
%within 60 seconds whenever the primary power supply fails to
%provide the minimum voltage required for proper operation.
%10.5.7.3 Required signals shall not be lost, interrupted, or
%delayed by more than 10 seconds as a result of the primary
%power failure.
%
%10.5.7.3.1 Storage batteries dedicated to the system or UPS
%arranged in accordance with the provisions of NFPA 111, Stan-
%dard on Stored Electrical Energy Emergency and Standby Power Sys-
%tems, shall be permitted to supplement the secondary power sup-
%ply to ensure required operation during the transfer period.
%10.5.7.3.2 Where a UPS is employed in 10.5.7.3.1, a positive
%means for disconnecting the input and output of the UPS sys-
%tem while maintaining continuity of power supply to the load
%shall be provided.
%











\subsection{Recomendaciones de instalación según NFPA 72}


Según el articulo 17.7.1.8 de la NFPA 72 los detectores de humo no se deben instalar bajo las siguientes condiciones:


\begin{itemize}
	
	\item Temperaturas por debajo de los 0°C.
	
	\item Temperaturas por encima de los 38°C.
	
	\item Humedad relativa al rededor del 93\%.
	
	\item Velocidades del viento superiores a 1.5 m/s.
	
\end{itemize}







%17.14.6 Manual fire alarm boxes shall be located within 60 in.
%(1.52 m) of the exit doorway opening at each exit on each floor.
%17.14.7 Manual fire alarm boxes shall be mounted on both
%sides of grouped openings over 40 ft (12.2 m) in width, and
%within 60 in. (1.52 m) of each side of the opening.










\subsection{Cálculo del calibre de cable según NFPA 70}





\subsection{Tipos de cableado según NFPA 72}

%12.3* Pathway Class Designations. Pathways shall be desig-
%nated as Class A, Class B, Class C, Class D, Class E, or Class X,
%depending on their performance.
%
%12.3.1* Class A. A pathway shall be designated as Class A when
%it performs as follows:
%(1) It includes a redundant path.
%(2) Operational capability continues past a single open.
%(3) Conditions that affect the intended operation of the path
%are annunciated.
%
%12.3.2* Class B. A pathway shall be designated as Class B when
%it performs as follows:
%(1) It does not include a redundant path.
%(2) Operational capability stops at a single open.
%(3) Conditions that affect the intended operation of the path
%are annunciated.
%
%12.3.3* Class C. A pathway shall be designated as Class C when
%it performs as follows:
%(1) It includes one or more pathways where operational capa-
%bility is verified via end-to-end communication, but the
%integrity of individual paths is not monitored.
%(2) A loss of end-to-end communication is annunciated.
%
%12.3.4* Class D. A pathway shall be designated as Class D when
%it has fail-safe operation, where no fault is annunciated, but
%the intended operation is performed in the event of a pathway
%failure.
%
%12.3.5* Class E. A pathway shall be designated as Class E when
%it is not monitored for integrity.
%
%12.3.6* Class X. A pathway shall be designated as Class X when
%it performs as follows:
%(1) It includes a redundant path.
%(2) Operational capability continues past a single open or
%short-circuit.
%(3) Conditions that affect the intended operation of the path
%are annunciated.



\newpage

\section{Sistema de alarmas contra intrusión}


\subsection{Dispositivos de detección de intrusión}

\subsection{Dispositivos de notificación de intrusión}

\subsection{Recomendaciones de diseño según CIEMI}


\newpage


\section{Sistema de control de acceso}


\subsection{Dispositivos de control de acceso}

\subsection{Recomendaciones de diseño según CIEMI}


\newpage


\section{Sistema de CCTV IP}


\subsection{Dispositivos de CCTV IP}

\subsection{Recomendaciones de diseño según TIA}


\newpage


\section{Sistema de iluminación}


\subsection{Estudio de iluminación}

\subsection{Recomendaciones de niveles de iluminación según INTECO}

\subsection{Tipos de iluminación y luminarias}

\subsection{Tipos de aislamiento de cable según NFPA 70}

\subsection{Cálculo del calibre de cable en circuitos ramales según NFPA 70}

\subsection{Cálculo protecciones en circuitos ramales según NFPA 70}

















