% ----------------------
  \chapter{Marco Teórico} 
% ----------------------
\label{C:antecedentes}

\section{Sistema de alarmas contra incendio}


\subsection{Tipos de fuego}

Cuando ocurre un incendio este se debe a la combustión de ciertos materiales, razón por la cual según la NFPA 72 y el cuerpo de bomberos de Costa Rica clasifica los tipos de fuego en: \cite{Bomberos}\\

\begin{itemize}
	
	\item \textbf{Clase A:} 
	
\begin{itemize}
	\item \textbf{Material combustible:} combustibles comunes como madera, tela, papel, caucho y plásticos.
\end{itemize}
	
	
	\item \textbf{Clase B:} 

\begin{itemize}
	\item \textbf{Material combustible:} líquidos y gases inflamables como aceites, grasas,
	alquitranes, base de pinturas y lacas.
\end{itemize}	


	\item \textbf{Clase C:} 

\begin{itemize}
	\item \textbf{Material combustible:} equipos eléctricos energizados.
\end{itemize}



	\item \textbf{Clase D:} 

\begin{itemize}
	\item \textbf{Material combustible:} metales como magnesio, titanio, zirconio, sodio, litio, potasio entre otros.
\end{itemize}

	\item \textbf{Clase K:} 

\begin{itemize}
	\item \textbf{Material combustible:} utensilios o materiales de cocina como aceites minerales, animales y grasas.
\end{itemize}

	
	
\end{itemize}


\subsection{Dispositivos de iniciación ó detección de incendio}


Son los dispositivos del sistema que se encargan de censar constantemente las condiciones del ambiente en busca de indicios de incendio. Estos dispositivos o sensores pueden ser direccionales o no, es decir que pueden indicar la locación o zona del incendio o bien solo activar el sistema. El fuego tiene ciertas características que pueden ser censadas y dar indicios de incendio, tales como humo, llama, temperatura y calor. 


\subsubsection{Detector de humo}







%
%3.3.59.1 Air Sampling–Type Detector. A detector that con-
%sists of a piping or tubing distribution network that runs
%from the detector to the area(s) to be protected. An aspira-
%tion fan in the detector housing draws air from the pro-
%tected area back to the detector through air sampling
%ports, piping, or tubing. At the detector, the air is analyzed
%for fire products. (SIG-IDS)
%3.3.59.2 Automatic Fire Detector. A device designed to de-
%tect the presence of a fire signature and to initiate action.
%For the purpose of this Code, automatic fire detectors are
%classified as follows: Automatic Fire Extinguishing or Sup-
%pression System Operation Detector, Fire–Gas Detector,
%Heat Detector, Other Fire Detectors, Radiant Energy–
%Sensing Fire Detector, Smoke Detector. (SIG-IDS)
%3.3.59.3 Automatic Fire Extinguishing or Suppression System
%Operation Detector. A device that automatically detects the
%operation of a fire extinguishing or suppression system by
%means appropriate to the system employed. (SIG-IDS)
%3.3.59.4* Combination Detector. A device that either re-
%sponds to more than one of the fire phenomena or em-
%ploys more than one operating principle to sense one of
%these phenomena. Typical examples are a combination of a
%heat detector with a smoke detector or a combination rate-
%of-rise and fixed-temperature heat detector. This device
%has listings for each sensing method employed. (SIG-IDS)
%3.3.59.5 Electrical Conductivity Heat Detector. A line-type or
%spot-type sensing element in which resistance varies as a
%function of temperature. (SIG-IDS)
%3.3.59.6 Fire–Gas Detector. A device that detects gases pro-
%duced by a fire. (SIG-IDS)
%3.3.59.7* Fixed-Temperature Detector. A device that responds
%when its operating element becomes heated to a predeter-
%mined level. (SIG-IDS)
%3.3.59.8* Flame Detector. A radiant energy–sensing fire de-
%tector that detects the radiant energy emitted by a flame.
%(Refer to A.17.8.2.) (SIG-IDS)
%3.3.59.9 Heat Detector. A fire detector that detects either
%abnormally high temperature or rate of temperature rise,
%or both. (SIG-IDS)
%3.3.59.10 Line-Type Detector. A device in which detection is
%continuous along a path. Typical examples are rate-of-rise
%pneumatic tubing detectors, projected beam smoke detec-
%tors, and heat-sensitive cable. (SIG-IDS)
%3.3.59.11* Multi-Criteria Detector. A device that contains
%multiple sensors that separately respond to physical stimu-
%lus such as heat, smoke, or fire gases, or employs more than
%one sensor to sense the same stimulus. This sensor is ca-
%pable of generating only one alarm signal from the sensors
%employed in the design either independently or in combi-
%nation. The sensor output signal is mathematically evalu-
%ated to determine when an alarm signal is warranted. The
%evaluation can be performed either at the detector or at
%the control unit. This detector has a single listing that es-
%tablishes the primary function of the detector. (SIG-IDS)
%3.3.59.12* Multi-Sensor Detector. A device that contains
%multiple sensors that separately respond to physical stimu-
%lus such as heat, smoke, or fire gases, or employs more than
%one sensor to sense the same stimulus. A device capable of
%generating multiple alarm signals from any one of the sen-
%sors employed in the design, independently or in combina-
%tion. The sensor output signals are mathematically evalu-
%ated to determine when an alarm signal is warranted. The
%evaluation can be performed either at the detector or at
%the control unit. This device has listings for each sensing
%method employed. (SIG-IDS)
%3.3.59.13 Other Fire Detectors. Devices that detect a phe-
%nomenon other than heat, smoke, flame, or gases pro-
%duced by a fire. (SIG-IDS)
%3.3.59.14 Pneumatic Rate-of-Rise Tubing Heat Detector. A
%line-type detector comprising small-diameter tubing, usu-
%ally copper, that is installed on the ceiling or high on the
%walls throughout the protected area. The tubing is termi-
%nated in a detector unit containing diaphragms and associ-
%ated contacts set to actuate at a predetermined pressure.
%The system is sealed except for calibrated vents that com-
%pensate for normal changes in temperature. (SIG-IDS)
%3.3.59.15 Projected Beam–Type Detector. A type of photo-
%electric light obscuration smoke detector wherein the
%beam spans the protected area. (SIG-IDS)
%3.3.59.16 Radiant Energy–Sensing Fire Detector. A device
%that detects radiant energy, such as ultraviolet, visible, or
%infrared, that is emitted as a product of combustion reac-
%tion and obeys the laws of optics. (SIG-IDS)
%3.3.59.17* Rate Compensation Detector. A device that re-
%sponds when the temperature of the air surrounding the
%device reaches a predetermined level, regardless of the rate
%of temperature rise. (SIG-IDS)
%3.3.59.18* Rate-of-Rise Detector. A device that responds
%when the temperature rises at a rate exceeding a predeter-
%mined value. (SIG-IDS)
%3.3.59.19 Smoke Detector. A device that detects visible or
%invisible particles of combustion. (SIG-IDS)
%3.3.59.20 Spark/Ember Detector. A radiant energy–sensing
%fire detector that is designed to detect sparks or embers, or
%both. These devices are normally intended to operate in
%dark environments and in the infrared part of the spec-
%trum. (SIG-IDS)
%3.3.59.21 Spot-Type Detector. A device in which the detect-
%ing element is concentrated at a particular location. Typi-
%cal examples are bimetallic detectors, fusible alloy detec-
%tors, certain pneumatic rate-of-rise detectors, certain
%smoke detectors, and thermoelectric detectors. (SIG-IDS)
%
%
%
%3.3.252.1 Cloud Chamber Smoke Detection. The principle of
%using an air sample drawn from the protected area into a
%high-humidity chamber combined with a lowering of
%chamber pressure to create an environment in which the
%resultant moisture in the air condenses on any smoke par-
%ticles present, forming a cloud. The cloud density is mea-
%sured by a photoelectric principle. The density signal is
%processed and used to convey an alarm condition when it
%meets preset criteria. (SIG-IDS)
%3.3.252.2* Ionization Smoke Detection. The principle of us-
%ing a small amount of radioactive material to ionize the air
%between two differentially charged electrodes to sense the
%presence of smoke particles. Smoke particles entering the
%ionization volume decrease the conductance of the air by
%reducing ion mobility. The reduced conductance signal is
%processed and used to convey an alarm condition when it
%meets preset criteria. (SIG-IDS)
%3.3.252.3* Photoelectric Light Obscuration Smoke Detection.
%The principle of using a light source and a photosensitive
%sensor onto which the principal portion of the source emis-
%sions is focused. When smoke particles enter the light path,
%some of the light is scattered and some is absorbed, thereby
%reducing the light reaching the receiving sensor. The light
%reduction signal is processed and used to convey an alarm
%condition when it meets preset criteria. (SIG-IDS)
%3.3.252.4* Photoelectric Light-Scattering Smoke Detection.
%The principle of using a light source and a photosensitive
%sensor arranged so that the rays from the light source do
%not normally fall onto the photosensitive sensor. When
%smoke particles enter the light path, some of the light is
%scattered by reflection and refraction onto the sensor. The
%light signal is processed and used to convey an alarm con-
%dition when it meets preset criteria. (SIG-IDS)
%3.3.252.5* Video Image Smoke Detection (VISD). The prin-
%ciple of using automatic analysis of real-time video images
%to detect the presence of smoke. (SIG-IDS)












\subsection{Dispositivos de notificación de incendio}

%3.3.160 Notification Appliance. A fire alarm system compo-
%nent such as a bell, horn, speaker, light, or text display that
%provides audible, tactile, or visible outputs, or any combina-
%tion thereof. (SIG-NAS)
%3.3.160.1 Audible Notification Appliance. A notification ap-
%pliance that alerts by the sense of hearing. (SIG-NAS)
%3.3.160.1.1 Exit Marking Audible Notification Appliance. An
%audible notification appliance that marks building exits
%and areas of refuge by the sense of hearing for the purpose
%of evacuation or relocation. (SIG-NAS)
%3.3.160.1.2* Textual Audible Notification Appliance. A notifi-
%cation appliance that conveys a stream of audible informa-
%tion. (SIG-NAS)
%3.3.160.2 Tactile Notification Appliance. A notification appli-
%ance that alerts by the sense of touch or vibration. (SIG-NAS)
%3.3.160.3 Visible Notification Appliance. A notification ap-
%pliance that alerts by the sense of sight. (SIG-NAS)
%3.3.160.3.1 Textual Visible Notification Appliance. A notifica-
%tion appliance that conveys a stream of visible information
%that displays an alphanumeric or pictorial message. Textual
%visible notification appliances provide temporary text, per-
%manent text, or symbols. Textual visible notification appli-
%ances include, but are not limited to, annunciators, moni-
%tors, CRTs, displays, and printers. (SIG-NAS)
%
%
%
%
%
%
%10.5.6.3* Capacity.
%
%10.5.6.3.1 The secondary power supply shall have sufficient
%capacity to operate the system under quiescent load (system
%operating in a nonalarm condition) for a minimum of
%24 hours and, at the end of that period, shall be capable of
%operating all alarm notification appliances used for evacuation
%or to direct aid to the location of an emergency for 5 minutes,
%unless otherwise permitted or required by the following:
%
%(1) Battery calculations shall include a 20 percent safety mar-
%gin to the calculated amp-hour rating.
%(2) The secondary power supply for in-building fire emer-
%gency voice/alarm communications service shall be ca-
%pable of operating the system under quiescent load for a
%minimum of 24 hours and then shall be capable of oper-
%ating the system during a fire or other emergency condi-
%tion for a period of 15 minutes at maximum connected
%load.
%(3) The secondary power supply capacity for supervising sta-
%tion facilities and equipment shall be capable of support-
%ing operations for a minimum of 24 hours.
%(4) The secondary power supply for high-power speaker ar-
%rays used for wide-area mass notification systems shall be
%in accordance with 24.4.3.4.2.2.
%(5) The secondary power supply for textual visible appliances
%shall be in accordance with 24.4.3.4.7.1.
%(6) The secondary power supply capacity for central control sta-
%tions of a wide-area mass notification systems shall be ca-
%pable of supporting operations for a minimum of 24 hours.
%(7) The secondary power supply for in-building mass notifica-
%tion systems shall be capable of operating the system un-
%der quiescent load for a minimum of 24 hours and then
%shall be capable of operating the system during emer-
%gency condition for a period of 15 minutes at maximum
%connected load.
%10.5.6.3.2 The secondary power supply capacity required
%shall include all power supply loads that are not automatically
%disconnected upon the transfer to secondary power supply.
%10.5.6.4 Secondary Power Operation.
%10.5.6.4.1 Operation on secondary power shall not affect the
%required performance of a system or supervising station facility,
%including alarm, supervisory, and trouble signals and indications.
%10.5.6.4.2 Systems operating on secondary power shall com-
%ply with Section 10.17.
%10.5.6.4.3 While operating on secondary power, audio ampli-
%fier monitoring shall comply with 10.17.2.1.2.
%10.5.7* Continuity of Power Supplies.
%10.5.7.1 The secondary power supply shall automatically pro-
%vide power to the protected premises system within 10 seconds
%whenever the primary power supply fails to provide the mini-
%mum voltage required for proper operation.
%10.5.7.2 The secondary power supply shall automatically pro-
%vide power to the supervising station facility and equipment
%within 60 seconds whenever the primary power supply fails to
%provide the minimum voltage required for proper operation.
%10.5.7.3 Required signals shall not be lost, interrupted, or
%delayed by more than 10 seconds as a result of the primary
%power failure.
%
%10.5.7.3.1 Storage batteries dedicated to the system or UPS
%arranged in accordance with the provisions of NFPA 111, Stan-
%dard on Stored Electrical Energy Emergency and Standby Power Sys-
%tems, shall be permitted to supplement the secondary power sup-
%ply to ensure required operation during the transfer period.
%10.5.7.3.2 Where a UPS is employed in 10.5.7.3.1, a positive
%means for disconnecting the input and output of the UPS sys-
%tem while maintaining continuity of power supply to the load
%shall be provided.
%











\subsection{Recomendaciones de instalación según NFPA 72}



\subsection{Cálculo del calibre de cable según NFPA 72}





\subsection{Tipos de cableado según NFPA 72}




\newpage

\section{Sistema de alarmas contra intrusión}


\subsection{Dispositivos de detección de intrusión}

\subsection{Dispositivos de notificación de intrusión}

\subsection{Recomendaciones de diseño según CIEMI}


\newpage


\section{Sistema de control de acceso}


\subsection{Dispositivos de control de acceso}

\subsection{Recomendaciones de diseño según CIEMI}


\newpage


\section{Sistema de CCTV IP}


\subsection{Dispositivos de CCTV IP}

\subsection{Recomendaciones de diseño según TIA}


\newpage


\section{Sistema de iluminación}


\subsection{Tipos de aislamiento de cable según NFPA 70}

\subsection{Cálculo del calibre de cable en circuitos ramales según NFPA 70}

\subsection{Cálculo protecciones en circuitos ramales según NFPA 70}

\subsection{Tipos de iluminación y luminarias}

\subsection{Recomendaciones de niveles de iluminación según INTECO}
















