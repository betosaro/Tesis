% ----------------------
  \chapter{Introducción}
% ----------------------
\label{C:introduccion}

Este proyecto nace de la inquietud de los encargados del Planetario de la Universidad de Costar Rica respecto a deficiencias en el sistema de iluminación. Además al estudiar más detalladamente el proyecto se pudo constatar notables carencias y deficiencias en los sistemas de seguridad humana como lo son el sistema de detección de incendios, el sistema de alarmas contra robo y la ausencia de un sistema de vigilancia de circuito cerrado de televisión (CCTV). Debido a esto, se propone realizar un diseño eléctrico de este edificio ajustado a las necesidades reales de los usuarios, de modo que represente la base para una futura remodelación eléctrica del mismo. Dicho diseño consiste en la elaboración de los cálculos, planos y especificaciones técnicas necesarias para que cada uno de los sistemas diseñados funcione apropiadamente.

%- sección ----------------------------------------------------------


\section{Alcance del proyecto}

El alcance de este proyecto se limita específicamente al diseño de una readecuación de los sistemas de iluminación y de seguridad humana del planetario de la Universidad de Costa Rica, ya que se considera que el resto de los sistemas actualmente en funcionamiento fueron dimensionados apropiadamente, por lo cual propician disconformidad en los usuarios. Por último se propone la realización de un cambio en los modelos de los tomacorrientes por un aspecto meramente estético, pero no es parte del alcance de este proyecto la reubicación, ni el cálculo de protecciones, ni el dimensionamiento del cableado y la canalización para este sistema.

\section{Objetivos}

\vspace{0.4cm}

\subsection{Objetivo general}
Diseñar los planos eléctricos y especificaciones técnicas necesarios para satisfacer las necesidades actuales en materia eléctrica y de seguridad humana del Planetario de la Universidad de Costa Rica en una futura remodelación. 

\subsection{Objetivos específicos}
Para el desarrollo de este proyecto se establecieron los siguientes objetivos:

\begin{enumerate}
	\item Realizar un levantamiento de la condición eléctrica actual del edificio.
	\item Realizar un estudio de iluminación del edificio.
	\item Elegir los equipos eléctricos y de seguridad humana que mejor se ajusten a las necesidades del proyecto.
	\item Diseñar la ubicación de los equipos eléctricos y de seguridad humana que se van a instalar en la edificación.
	\item Realizar un estudio de la carga a instalar.
	\item Realizar los cálculos necesarios para que los sistemas diseñados funcionen correctamente.
	\item Elaborar los planos constructivos en formato .DWG.
	\item Elaborar las especificaciones técnicas del proyecto.
\end{enumerate}

\section{Metodología}
La metodología utilizada debe listarse en forma cronológica.

El desarrollo del trabajo incluyó los siguientes pasos y procedimientos, listados en secuencia:

\begin{enumerate}
	\item Solicitud de los planos eléctricos actuales del planetario de la Universidad de Costa Rica a la Oficina Ejecutora del Programa de Inversiones de la Universidad de Costa Rica (OEPI).
	\item Revisión de sitio contra planos suministrados por la OEPI sobre la condición eléctrica actual del edificio.
	\item Realización de un estudio de iluminación del edificio mediante el uso del software DIALux considerando los pasillos como sala de exhibición.
	\item Elección de los modelos de luminarias más adecuados utilizando como referencia los valores de luminosidad proporcionados por el software DIALux y comparándolos con los modelos de luminarias LED del catálogo 2017 de Sylvania.
	\item Realización de una pequeña investigación acerca de los estándares y marcas de equipos de seguridad humana que utiliza la Universidad de Costa Rica.
	\item Elección de los equipos de seguridad humana que mejor se ajusten a las necesidades del proyecto tomando en cuenta los estándares y marcas que utiliza la Universidad de Costa Rica.
	\item Elaboración de una propuesta y dibujo en formato .DWG de la ubicación de las luminarias y equipos de seguridad humana que se van a instalar en la edificación.
	\item Realización un estudio de la carga eléctrica nueva a instalar, tomando en cuenta la que se eliminará.
	\item Elaboración del cálculo de las protecciones y dimensionamiento de cableado y canalización necesarios para la instalación de los equipos nuevos. Se incluye un nuevo cálculo de las acometidas en caso de ser necesaria su sustitución.
	\item Realización de una reubicación de los equipos propuestos a instalar en caso de tener que eliminar algunos por condiciones de carga eléctrica.
	\item Elaboración final de los planos constructivos en formato .DWG incluyendo las ubicaciones finales de los equipos.
	\item Redacción las especificaciones técnicas del proyecto, incluyendo las marcas, modelos y certificaciones permitidas para los materiales, métodos y condiciones de instalación de dichos equipos. Se agregará toda aquella información que se considere necesaria para la óptima conclusión del proyecto.
\end{enumerate}

%\section{Desarrollo o contenido}
%Como último párrafo de la introducción, debe indicarse al lector lo que se presenta en el informe, con una breve descripción del contenido de cada capítulo, mostrando la secuencia lógica de estos.

