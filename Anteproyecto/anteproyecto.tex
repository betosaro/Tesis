% ------------------------------------
% UNIVERSIDAD DE COSTA RICA
% Facultad de Ingeniería
% Escuela de Ingeniería Eléctrica
% IE0499 - Proyecto Eléctrico
%
% PLANTILLA Y GUÍA DEL ANTEPROYECTO
% Versión: v0.1 (marzo 2017)
% ------------------------------------

\documentclass{anteproyecto}
% Opción [vacio] para limpiar los espacios

% Título del proyecto
\titulo{Diseño de una readecuación eléctrica del edificio del Planetario de la Universidad de Costa Rica}

% Estudiante
\autor{Luis Alberto Salazar Romero}
\carne{B36359}
\email{betosaro10@gmail.com}
\telefono{8856-9893}

% Profesor(a) guía (nombre y correo electrónico)
\guia{Ing. Irene Víquez Barrantes}
\eguia{irene.viquez@ucr.ac.cr}

% Profesores lectores (nombre y correo electrónico)
\lectorA{Ing. Osvaldo Fernandez Cascante}
\electorA{osvaldo.fernandez\_c@ucr.ac.cr}
\lectorB{Ing. Jorge Sanchez Monge}
\electorB{jorge.sanchez@ucr.ac.cr}

% Semestre
\sem{I}		% I o II
\ano{2018}	% Formato AAAA

%%%%%%%%%%%%%%%%
\begin{document}
%%%%%%%%%%%%%%%%

\encabezado

\section*{Datos generales}
%-------------------------

\subsection*{Estudiante}
\estudiante

\subsection*{Profesor guía}
\profesorguia

\subsection*{Profesores lectores}
\profesoreslectores

\section*{Datos del proyecto}
%----------------------------

\subsection*{Título}
\eltitulo

\newpage

\subsection*{Descripción}

\ifvacio
	\vspace{4cm}
\else
% Descripción --------<<<
Este proyecto nace de la inquietud de los encargados del Planetario de la Universidad de Costar Rica respecto a deficiencias en el sistema de iluminación. Además al estudiar más detalladamente el proyecto se pudo constatar notables carencias y deficiencias en los sistemas de seguridad humana. Debido a esto, se propone realizar un diseño eléctrico de este edificio ajustado a las necesidades reales de los usuarios, de modo que represente la base para una futura remodelación eléctrica del mismo. Dicho diseño consiste en la elaboración de los cálculos, planos y especificaciones técnicas necesarias para que cada uno de los sistemas diseñados funcione apropiadamente. 

% --------------------<<<
\fi

\subsection*{Objetivo general}
\ifvacio
\vspace{4cm}
\else
% Objetivo general ----------<<<
Diseñar los planos eléctricos y especificaciones técnicas necesarios para satisfacer las necesidades actuales en materia eléctrica y de seguridad humana del Planetario de la Universidad de Costa Rica en una futura remodelación.
% --------------------<<<
\fi

\subsection*{Objetivos específicos}
\ifvacio
	\vspace{4cm}
\else
% Objetivos ----------<<<
\begin{enumerate}
\item Realizar un levantamiento de la condición eléctrica actual del edificio.
\item Realizar un estudio de iluminación del edificio.
\item Elegir los equipos eléctricos y de seguridad humana que mejor se ajusten a las necesidades del proyecto.
\item Diseñar la ubicación de los equipos eléctricos y de seguridad humana que se van a instalar en la edificación.
\item Realizar un estudio de la carga a instalar.
\item Realizar los cálculos necesarios para que los sistemas diseñados funcionen correctamente.
\item Elaborar los planos constructivos en formato DWG.
\item Elaborar las especificaciones técnicas del proyecto.
\end{enumerate}
% --------------------<<<
\fi

\subsection*{Clasificación temática}
\ifvacio
	\vspace{1cm}
\else
% \fbox{Temas}
\fbox{Diseño eléctrico} \fbox{Sistemas de potencia} \fbox{Seguridad humana} \fbox{Iluminación}
\fi

\section*{Declaración de último semestre}

% Opción correcta: \seleccionar{*}
\begin{itemize}
\item[\seleccionar{*}] Sí, este es mi último semestre.
\item[\seleccionar{}] No, este no es mi último semestre.

\end{itemize}

\section*{Firmas}
%----------------

\firmas


%%%%%%%%%%%%%%
\end{document}
%%%%%%%%%%%%%%
